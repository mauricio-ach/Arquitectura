
\documentclass[12pt, letterpaper, twoside]{article}
\usepackage[utf8]{inputenc}
\usepackage{enumerate} % enumerados
\usepackage{multirow} % para las tablas
\usepackage{float} % para usar [H]

\title{Organizaci\'on y Arquitectura de computadoras Práctica 1}
\author{Araujo Chávez Mauricio}
\date{24 de Agosto de 2017}
 
\begin{document}

\maketitle

Datos de la computadora:

\begin{itemize}
\item Fabricante y modelo de la computadora:
  
  Hewlett Packard, HP Notebook .
  
\item Fabricante, modelo, frecuencia, número de núcleos, y arquitectura del procesador:
  
  Intel Pentium N3540, 2.67GHz, 4 núcleos, 64 bits.
  
\item Capacidad de memoria RAM y de cachés de los procesadores:
  
  4 GB Memoria RAM, 2 MB memoria caché del procesador.
    
\item Capacidad del disco duro:
  
  1 TB de capacidad.
  
\item Distribución de linux y versión del kernel:
  
  Ubuntu 17.04, Kernel: 4.10.0-32-generic. 
\end{itemize}

\begin{table}[H]
\begin{center}
\begin{tabular}{|l|l|}
\hline
Nombre de la prueba & Resultado \\
\hline \hline
GZIP Compression & 27.28 Seconds \\ \hline
DCRAW & 191.74 Seconds \\ \hline
FLAC Audio Encoding & 23.90 Seconds \\ \hline
REDIS & 386115.21 Requests Per Second \\ \hline
Timed MrBayes Analysis & 81.41 Seconds \\ \hline
Timed M Player Compilation & 210.01 Seconds \\ \hline
Timed PHP Compilation & 125.24 Seconds\\ \hline
\end{tabular}
\caption{Resultado de las pruebas}
\label{tabla:sencilla}
\end{center}
\end{table}

\begin{table}[H]
\begin{center}
\begin{tabular}{|l|l|}
\hline
Nombre de la prueba & Resultado \\
\hline \hline
GZIP Compression & 26.4 Seconds \\ \hline
DCRAW & 136.38 Seconds \\ \hline
FLAC Audio Encoding & 20.00 Seconds \\ \hline
REDIS & 416814.34 Requests Per Second \\ \hline
Timed MrBayes Analysis & 42.39 Seconds \\ \hline
Timed M Player Compilation & 233.62 Seconds \\ \hline
Timed PHP Compilation & 117.00 Seconds\\ \hline
\end{tabular}
\caption{Resultado de las pruebas(2a Computadora)}
\label{tabla:sencilla}
\end{center}
\end{table}

\begin{table}[H]
\begin{center}
\begin{tabular}{|l|l|}
\hline
Nombre de la prueba & Resultado \\
\hline \hline
GZIP Compression & 60.69 Seconds \\ \hline
DCRAW & 108.98 Seconds \\ \hline
FLAC Audio Encoding & 14.00 Seconds \\ \hline
REDIS & 752900.60 Requests Per Second \\ \hline
Timed MrBayes Analysis & 95.70 Seconds \\ \hline
Timed M Player Compilation & 226.43 Seconds \\ \hline
Timed PHP Compilation & 131.60 Seconds\\ \hline
\end{tabular}
\caption{Resultado de las pruebas(3a Computadora)}
\label{tabla:sencilla}
\end{center}
\end{table}

\begin{table}[H]
\begin{center}
\begin{tabular}{|l|l|}
\hline
Nombre de la prueba & Resultado \\
\hline \hline
GZIP Compression & 40.53 Seconds \\ \hline
DCRAW & 167.28 Seconds \\ \hline
FLAC Audio Encoding & 32.02 Seconds \\ \hline
REDIS & 166265.13 Requests Per Second \\ \hline
Timed MrBayes Analysis & 78.785 Seconds \\ \hline
Timed M Player Compilation & 295.53 Seconds \\ \hline
Timed PHP Compilation & 142.85 Seconds\\ \hline
\end{tabular}
\caption{Resultado de las pruebas(4a computadora)}
\label{tabla:sencilla}
\end{center}
\end{table}

\begin{table}[H]
\begin{center}
\begin{tabular}{|l|l|}
\hline
Nombre de la prueba & Propósito \\
\hline \hline
GZIP Compression &  Tiempo de respuesta\\ \hline
DCRAW & Tiempo de respuesta \\ \hline
FLAC Audio Encoding & Tiempo de respuesta \\ \hline
REDIS & Rendimiento \\ \hline
Timed MrBayes Analysis & Tiempo de respuesta \\ \hline
Timed M Player Compilation & Tiempo de respuesta \\ \hline
Timed PHP Compilation & Tiempo de respuesta \\ \hline
\end{tabular}
\caption{Tipo de pruebas}
\label{tabla:sencilla}
\end{center}
\end{table}

\begin{description}
  \item[Ejercicios]
\end{description}
Usando la medida de tendencia central adecuada y tu reporte de resultados, calcula:
\begin{itemize}
  \item La medida de tiempo de respuesta \\ \textbf{55.1758 segundos}
\end{itemize}
\begin{itemize}
  \item La medida de rendimiento \\ \textbf{386115.21 request per second}
\end{itemize}
Calcula los tiempos normalizados y obtén la medida de tendencia central adecuada para cada una de las computadoras.
\begin{itemize}
  \item \textit{Computadora 1 y 2:} \textbf{1.21} \\ \textit{Computadora 1 y 3:} \textbf{1.09} \\ \textit{Computadora 1 y 4:} \textbf{0.78} 
\end{itemize}
\begin{itemize}
  \item \textit{Computadora 2:} \textbf{45.5825 segundos} \\ \textit{Computadora 3:} \textbf{50.1897} \\ \textit{Computadora 4:} \textbf{70.6227 segundos}
\end{itemize} 
Plantea un caso de uso de computadora, de acuerdo a los requerimientos del usuario pondera los resultados y obten la medida de desempeño.
\begin{itemize}
  \item Suponemos que el ususario requiere un máximo tiempo de respuesta en compiladores y cifrados por lo tanto tendríamos: \\.25 en PHP test; .25 en MPlayer test; .25 en GZIP test; .10 en MrBayes test; .10 en DCRAW test y .5 en FLAC test. \\ De donde obtenemos los siguientes resultados: \\ \textit{Computadora 1:} \textbf{129.8975 segundos} \\ \textit{Computadora 2:} \textbf{122.132 segundos} \\ \textit{Computadora 3:} \textbf{132.148 segundos} \\ \textit{Computadora 4:} \textbf{160.344 segundos} \\ De donde podemos deducir que la computadora 2 sería la ideal para tales condiciones de trabajo.
\end{itemize}

\begin{description}
  \item[Preguntas] 
\end{description}
\begin{itemize}
  \item ¿Cuál computadora tiene el mejor tiempo de ejecución? Comparada con la peor medida, ¿Por qué factor es mejor la computadora? \\ \textit{El tiempo de ejecución de la computadora 2 es \textbf{0.64} veces más rápido que la computadora 4. Debido a la diferencia en sus frecuencias de reloj.}
\end{itemize}
\begin{itemize}
  \item ¿Cuál computadora tiene el mejor rendimiento? Comparada con la peor medida, ¿Por qué factor es mejor la computadora? \\ \textit{El rendimiento de la computadora 1 es \textbf{0.51} veces mejor que la computadora 3.}
\end{itemize}
\begin{itemize}
  \item De acuerdo a la computadora de referencia, ¿cuál computadora tiene el mejor desempeño y cuál computadora tiene el peor desempeño? \\ \textit{El mejor desempeño lo tiene la computadora referencia y el peor lo tiene la computadora 3.}
\end{itemize}
\begin{itemize}
  \item ¿Cuál computadora tiene el mejor desempeño para el usuario planteado en el caso de uso? \\ \textit{La computadora 2.}
\end{itemize}
\begin{itemize}
  \item De entre los atributos de cada máquina comparada, ¿cuáles resultan determinantes en la pérdida o ganancia de desempeño? \\ \textit{La frecuencia de reloj y los ciclos de reloj promedio por instrucción.}
\end{itemize}

\end{document}
\grid
