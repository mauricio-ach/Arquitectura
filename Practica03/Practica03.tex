\documentclass[a4paper,11pt]{article}
\usepackage[T1]{fontenc}
\usepackage[utf8]{inputenc}
\usepackage{lmodern}
\usepackage[spanish]{babel}

\title{Organización y Arquitectura de Computadoras\\Practica 03}
\author{Mauricio Araujo Chávez\\312210047}
\date{13/Septiembre/2017}
\begin{document}
\maketitle

Preguntas: 

\begin{itemize}
  \item ¿Cuál es el procedimiento a seguir para desarrollar un circuito que resuelva un problema que involucre lógica combinacional? \\ Se debe crear la función que interpréte el problema, posteriormente dar su tabla de verdad, una vez realizado esto, se procede a crear el mapa de \textit{Karnaugh} para realizar la minimización de mintérminos o maxtérminos, según sea el caso; y por último con éstos se puede realizar el circuito que resuelve dicho problema.
\end{itemize}
\begin{itemize}
  \item Si una función de conmutación se evalúa a más ceros que unos ¿es conveniente usar mintérminos o maxtérminos? ¿En el caso que se evalue a más unos? \\ Cuando se evalúa a mas ceros que unos se ocupan mintérminos y cuando son más unos que ceros se ocupan maxtérminos.
\end{itemize}
\begin{itemize}
  \item Analizando el trabajo realizado, ¿cuáles son los incovenientes de desarrollar los circuitos de forma manual? \\ Que se presta a ser propenso a errores, puesto que el desarrollo de éste puede llegar a ser confuso en cierta manera al no interpretarlo correctamente, y al utilizar la herramienta \textit{logisim} se vuelve más intuitivo y provee una facilidad para realizar pruebas al circuito construido.
\end{itemize}
\end{document}
