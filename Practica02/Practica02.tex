\documentclass[a4paper,11pt]{article}
\usepackage[T1]{fontenc}
\usepackage[utf8]{inputenc}
\usepackage{lmodern}
\usepackage[spanish]{babel}

\title{Organización y Arquitectura de Computadoras\\Practica 02}
\author{Mauricio Araujo Chávez\\312210047}
\date{27/Agosto/2017}
\begin{document}

\maketitle
\begin{itemize}
  \item Compilar usando el comando: \\\textit{gcc -o xxxx medias.c -lm} \\ - Donde \textit{xxxx} es el nombre que se le quiera dar al programa.
\end{itemize}
\begin{itemize}
  \item Ejecutar usando el comando: \\\textit{./xxxx} 'O' numeros \\ - Donde \textit{xxxx} es el nombre del programa \\- \textit{'O'} la opción a calcular que puede ser:\\ --- \textbf{'A'} para media aritmetica \\--- \textbf{'H'} para media armónica \\--- \textbf{'G'} para media geométrica. \\ - \textit{numeros} son los numeros a los cuales se les quiere calcular media separados por espacios. \\ Por ejemplo \textit{./medias A 1 2 3}
\end{itemize}
\end{document}
